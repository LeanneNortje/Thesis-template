% Page layout
\usepackage[left=3cm,right=2cm,top=2.5cm,bottom=2.5cm]{geometry}
\usepackage{afterpage}
\newcommand\blankpage{%
	\null
	\thispagestyle{empty}%
	\addtocounter{page}{-1}%
	\newpage}

% Subsections
\setcounter{secnumdepth}{3}
\setcounter{tocdepth}{3}

% Figures
\usepackage[margin=\the\parindent,small,bf,rm]{caption}
\usepackage{graphicx}
\usepackage{pdfpages}
\setlength{\abovecaptionskip}{8pt}  % spacing above and below captions

% Font and text
\usepackage[english,afrikaans]{babel}
\usepackage{microtype}
\usepackage{setspace}
\newcommand{\myemph}[1]{{\sffamily\bfseries#1}}
\sloppy
\onehalfspacing
\usepackage{enumitem}
\usepackage{listings,xfp}
\usepackage{anyfontsize}
\usepackage{bold-extra}
\usepackage{quotchap}

\renewcommand\labelitemi{\tinytonormal$\gg$}

\makeatletter
{\large % Capture font definitions of \small
	\xdef\f@size@large{\f@size}
	\xdef\f@baselineskip@large{\f@baselineskip}
	\Large % Capture font definitions for \normalsize
	\xdef\f@size@Large{\f@size}
	\xdef\f@baselineskip@Large{\f@baselineskip}
}
% Define new \smalltonormalsize font size
\newcommand{\largetoLarge}{%
	\fontsize
	{\fpeval{(\f@size@large+\f@size@Large)/2}}
	{\fpeval{(\f@baselineskip@large+\f@baselineskip@Large)/2}}%
	\selectfont
}
\makeatother
\makeatletter
{\tiny % Capture font definitions of \small
	\xdef\f@size@tiny{\f@size}
	\xdef\f@baselineskip@tiny{\f@baselineskip}
	\normalsize % Capture font definitions for \normalsize
	\xdef\f@size@normalsize{\f@size}
	\xdef\f@baselineskip@normalsize{\f@baselineskip}
}
% Define new \smalltonormalsize font size
\newcommand{\tinytonormal}{%
	\fontsize
	{\fpeval{(\f@size@tiny+\f@size@normalsize)/2}}
	{\fpeval{(\f@baselineskip@tiny+\f@baselineskip@normalsize)/2}}%
	\selectfont
}
\makeatother
\makeatletter
{\tiny % Capture font definitions of \small
	\xdef\f@size@tiny{\f@size}
	\xdef\f@baselineskip@tiny{\f@baselineskip}
	\normalsize % Capture font definitions for \normalsize
	\xdef\f@size@normalsize{\f@size}
	\xdef\f@baselineskip@normalsize{\f@baselineskip}
}
% Define new \smalltonormalsize font size
\newcommand{\othertinytonormal}{%
	\fontsize
	{\fpeval{(\f@size@tiny+\f@size@normalsize)*0.4}}
	{\fpeval{(\f@baselineskip@tiny+\f@baselineskip@normalsize)*0.4}}%
	\selectfont
}
\makeatother

% Headings
\usepackage{textcase}
\usepackage{mfirstuc}
\usepackage{titlecaps}
\usepackage{stringstrings}
\usepackage[raggedright,rm,bf]{titlesec}
\usepackage{titlecaps}
\titlelabel{\thetitle.\quad}
\titleformat{\chapter}[display]{\LARGE\rmfamily\raggedleft}{\scshape \chaptertitlename\ \thechapter}{3pt}{\titlerule\vspace{1.5ex} \scshape \Huge \centering}[\vspace{1.5ex}\titlerule]
\titlespacing*{\chapter}{0pt}{30pt}{40pt}  % remove spacing before chapter headings

\titleformat{\section}{\rmfamily\scshape\Large}{\thesection}{1em}{}
\titleformat{\subsection}{\rmfamily\scshape\large}{\thesubsection}{1em}{}
\titleformat{\subsubsection}{\rmfamily\scshape\normalsize}{\thesubsubsection}{1em}{}

\newcommand{\ShortInTextTitle}[1]{\textbf{\scshape \xcapitalisewords{#1}}}

% Table of contents
\usepackage{xcolor,titletoc}
\makeatletter
\let\originall@chapter\l@chapter
\def\l@chapter#1#2{\originall@chapter{{\rmfamily #1}}{#2}}
\makeatother
\let \savenumberline \numberline
\def \numberline#1{\savenumberline{#1.}}

% Nomenclaure
\usepackage{acro}
\newlength\myitemwidth
\setlength\myitemwidth{3cm}
\newlist{myacronymlist}{description}{1}
\setlist[myacronymlist]{
	labelindent = 0pt ,
	labelsep    = 0pt ,
	leftmargin  = \myitemwidth ,
	labelwidth  = \myitemwidth ,
	itemindent  = 0pt ,
	format      = \normalfont \bfseries 
}
\DeclareAcroListStyle{myliststyle}{list}{
	list = myacronymlist
}
\acsetup{only-used=false,list-long-format=\xcapitalisewords,list-style=myliststyle}
\DeclareAcronym{AE}{
	short = AE ,
	long  = autoencoder
}
\DeclareAcronym{CAE}{
	short = CAE,
	long  = correspondence autoencoder,
	alt = Correspondence Autoencoder
}
\DeclareAcronym{MCAE}{
	short = MCAE,
	long  = multimodal correspondence autoencoder,
	alt = Multimodal Correspondence Autoencoder
}
\DeclareAcronym{MTriplet}{
	short = MTriplet,
	long  = multimodal triplet network,
	alt = Multimodal Triplet Network
}
%\DeclareAcronym{FFNN}{
%	short = FFNN,
%	long  = feedforward neural network,
%	alt = Feedforward Neural Networks
%}
%\DeclareAcronym{CNN}{
%	short = CNN,
%	long  = convolutional neural network,
%	alt = Convolutional Neural Networks
%}
%\DeclareAcronym{RNN}{
%	short = RNN,
%	long  = recurrent neural network,
%	alt = Recurrent Neural Networks
%}
%\DeclareAcronym{MFCC}{
%	short = MFCC,
%	long  = mel-frequency cepstral coefficient
%}
%\DeclareAcronym{GRU}{
%	short = GRU,
%	long  = gated recurrent unit
%}
%\DeclareAcronym{DTW}{
%	short = DTW,
%	long  = dynamic time warping,
%	alt = Dynamic Time Warping
%}
%%\DeclareAcronym{AI}{
%%	short = AI,
%%	long  = artificial intelligence
%%}
%\DeclareAcronym{ML}{
%	short = ML,
%	long  = machine learning
%}
%\DeclareAcronym{DL}{
%	short = DL,
%	long  = deep learning
%}
%%\DeclareAcronym{DNN}{
%%	short = DNN,
%%	long  = deep neural network
%%}
%%\DeclareAcronym{ASR}{
%%	short = ASR,
%%	long  = automatic speech recognition
%%}
%\DeclareAcronym{HBPL}{
%	short = HBPL,
%	long  = hierarchical Bayesian program learning
%}
%%\DeclareAcronym{BPL}{
%%	short = BPL,
%%	long  = Bayesian program learning
%%}
%\DeclareAcronym{HHMM}{
%	short = HHMM,
%	long  = hierarchichal hidden Markov model
%}
%\DeclareAcronym{LSTM}{
%	short = LSTM, 
%	long = long short-term memory
%}
%\DeclareAcronym{SGD}{
%	short = SGD, 
%	long = stochastic gradient decent,
%	alt = Stochastic Gradient Decent
%}
%\DeclareAcronym{MAML}{
%	short = MAML, 
%	long = model-agnostic meta-learning
%}
%\DeclareAcronym{VAE}{
%	short = VAE, 
%	long = variational autoencoder
%}
\usepackage[intoc, english]{nomencl}

\renewcommand{\nomgroup}[1]{%
	\ifthenelse{\equal{#1}{V}}{\item[{\rmfamily\scshape\Large Variables}]}{%
		\ifthenelse{\equal{#1}{C}}{\item[{\rmfamily\scshape\Large Constants}]}{
			\ifthenelse{\equal{#1}{F}}{\item[{\rmfamily\scshape\Large Functions}]}{}}}%
}

\makenomenclature

\makeatletter
\renewcommand\tableofcontents{%
	\chapter*{\contentsname}%
	\@mkboth{\scshape \contentsname}%
	{\scshape \contentsname}%
	\@starttoc{toc}%
}
\renewcommand\listoftables{%
	\chapter*{\listtablename}%
	\@mkboth{\scshape\listtablename}%
	{\scshape\listtablename}%
	\@starttoc{lot}%
}
\renewcommand\listoffigures{%
	\chapter*{\listfigurename}%
	\@mkboth{\scshape\listfigurename}%
	{\scshape\listfigurename}%
	\@starttoc{lof}%
}
\makeatother


% Mathematics
\usepackage[cmex10]{amsmath}
\usepackage{amssymb}
\usepackage{cancel}
\DeclareMathOperator*{\argmax}{arg\,max}
\newcommand{\T}{^\textrm{T}}
\newcommand{\tr}{\textrm{tr}}
\renewcommand{\vec}[1]{\boldsymbol{\mathbf{#1}}}
\newcommand{\defeq}{\triangleq}

% Tables
\usepackage{placeins}
\usepackage{array}
\usepackage{booktabs,colortbl}
\usepackage{tabularx}
\usepackage{multirow}
\newcommand{\mytable}{
	\centering
	\small
	%     \renewcommand{\arraystretch}{1.2}
	\renewcommand{\arraystretch}{1.2}
}
\renewcommand{\tabularxcolumn}[1]{m{#1}}
\newcommand{\PreserveBackslash}[1]{\let\temp=\\#1\let\\=\temp}
\newcolumntype{C}{>{\centering\arraybackslash}X}
\newcolumntype{L}{>{\raggedright\arraybackslash}X}
\newcolumntype{A}[1]{>{\PreserveBackslash\raggedright}p{#1}}
\renewcommand{\topfraction}{.60}
\renewcommand{\bottomfraction}{.85}
\renewcommand{\textfraction}{.05}
\renewcommand{\floatpagefraction}{.66}
\renewcommand{\dbltopfraction}{.66}
\renewcommand{\dblfloatpagefraction}{.66}
\setcounter{topnumber}{2}
\setcounter{bottomnumber}{9}
\setcounter{totalnumber}{20}
\setcounter{dbltopnumber}{2}

% Descriptions
\setlist[description]{ 
	labelwidth=0.6\textwidth,%
	leftmargin=\labelwidth, 
	align=right,
	font=\normalfont\scshape\bfseries
}
\newcommand{\mydefinition}[2] {
	\begin{flushright}
		{\scshape\bfseries #1} #2 \\
	\end{flushright}
}

% Pseudo-code
\usepackage{algorithm}  % should go before \usepackage{hyperref}

% Table of contents and hyperlinks
\usepackage[linktocpage=true]{hyperref}
\hypersetup{colorlinks=true,linktoc=all,linkcolor=.,filecolor=.,urlcolor=.,citecolor=.}
\usepackage[nottoc,notlot,notlof]{tocbibind}
\usepackage{blindtext}

% Pseudo-code
\usepackage{algpseudocode}  % should go after \usepackage{hyperref}
\renewcommand{\thealgorithm}{\arabic{chapter}.\arabic{algorithm}} 
\captionsetup[algorithm]{labelfont={bf},font=small,labelsep=colon}

% Fix titlesec issue
\usepackage{etoolbox}
\makeatletter
\patchcmd{\ttlh@hang}{\parindent\z@}{\parindent\z@\leavevmode}{}{}
\patchcmd{\ttlh@hang}{\noindent}{}{}{}
\makeatother

% Bibliography
\usepackage[square,numbers,compress,sort]{natbib}
\bibliographystyle{IEEEtranN}
\newcommand{\mycite}[1]{\citeauthor{#1}'s~\cite{#1}}

% Header and footer
\newcommand{\listheaderfont}{\scshape}
\renewcommand{\tocetcmark}[1]{
	\markboth{\listheaderfont #1}{\listheaderfont #1}}
\usepackage{fancyhdr}
\usepackage{extramarks}
\pagestyle{fancy}
\fancyhf{}
\renewcommand{\chaptermark}[1]{\markboth{\normalsize \rmfamily \thechapter.\ \scshape #1}{\normalsize \rmfamily \thechapter.\ \scshape #1}}
\renewcommand{\sectionmark}[1]{\markright{\normalsize \rmfamily \thesection.\ \scshape #1}}
%\renewcommand{\sectionmark}[1]{\markright{\normalsize \rmfamily \thesection.\ \scshape #1}{}}
\fancyhead[C]{\color{oraclecolor} \rightmark \\ \color{rulecolor} \rule{\textwidth}{0.1pt}}
\fancyfoot{}
\fancyfoot[R]{\thepage}{}

\setlength\headheight{14.5pt}
\renewcommand{\headrulewidth}{0pt}
\fancypagestyle{plain}{\fancyhead{}
	\renewcommand{\headrulewidth}{0pt}
	%	\fancyfoot[C]{\thepage}
}

\newcommand{\mysection}[2]{
	%	\sectionmark{#2}
	\let\orisectionmark\sectionmark
	\renewcommand\sectionmark[1]{}%
	\section[#1]{#2 \orisectionmark{#2}}
	\orisectionmark{#2}
	\let\sectionmark\orisectionmark
}
\newcommand{\mychapter}[2]{
	%	\sectionmark{#2}
	\chapter[#1]{#2 \chaptermark{#2}}
}

% Miscellaneous
\interfootnotelinepenalty=1000
\newcommand*{\WaterMark}[2][0.2\paperwidth]{\AddToShipoutPicture*{\AtTextCenter{\parbox[c]{0pt}{\makebox[0pt][c]{\includegraphics[width=#1]{#2}}}}}}

% Colors and editing
\usepackage[prependcaption,textsize=tiny]{todonotes}
\setlength{\marginparwidth}{2.6cm}
\reversemarginpar
\definecolor{othercolor}{HTML}{CC0000}
\definecolor{oraclecolor}{HTML}{8B8589}
\definecolor{rulecolor}{HTML}{DCDCDC}
\definecolor{ashgrey}{rgb}{0.7, 0.75, 0.71}
\newcommand{\change}[1]{\textcolor{othercolor}{#1}}