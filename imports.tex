% Page layout
\usepackage[left=3cm,right=2cm,top=2.5cm,bottom=2.5cm]{geometry}
%\usepackage{cite}
\usepackage{afterpage}
\newcommand\blankpage{%
	\null
	\thispagestyle{empty}%
	\addtocounter{page}{-1}%
	\newpage}

% Subsections
\setcounter{secnumdepth}{3}
\setcounter{tocdepth}{3}

% Figures
\usepackage[margin=\the\parindent,small,bf,rm]{caption}
\usepackage{graphicx}
\usepackage{pdfpages}
\setlength{\abovecaptionskip}{8pt}  % spacing above and below captions
\usepackage{subfig}

% Font and text
\usepackage[english,afrikaans]{babel}
\usepackage{microtype}
\usepackage{setspace}
\newcommand{\myemph}[1]{{\sffamily\bfseries#1}}
\sloppy
\onehalfspacing
\usepackage{enumitem}
%\setlist[enumerate]{noitemsep, topsep=0pt, partopsep=0pt}
\usepackage{listings,xfp}
\usepackage{anyfontsize}
\usepackage{bold-extra}
\usepackage{quotchap}

\renewcommand\labelitemi{\tinytonormal$\divideontimes$}

\makeatletter
{\large % Capture font definitions of \small
	\xdef\f@size@large{\f@size}
	\xdef\f@baselineskip@large{\f@baselineskip}
	\Large % Capture font definitions for \normalsize
	\xdef\f@size@Large{\f@size}
	\xdef\f@baselineskip@Large{\f@baselineskip}
}
% Define new \smalltonormalsize font size
\newcommand{\largetoLarge}{%
	\fontsize
	{\fpeval{(\f@size@large+\f@size@Large)/2}}
	{\fpeval{(\f@baselineskip@large+\f@baselineskip@Large)/2}}%
	\selectfont
}
\makeatother
\makeatletter
{\tiny % Capture font definitions of \small
	\xdef\f@size@tiny{\f@size}
	\xdef\f@baselineskip@tiny{\f@baselineskip}
	\normalsize % Capture font definitions for \normalsize
	\xdef\f@size@normalsize{\f@size}
	\xdef\f@baselineskip@normalsize{\f@baselineskip}
}
% Define new \smalltonormalsize font size
\newcommand{\tinytonormal}{%
	\fontsize
	{\fpeval{(\f@size@tiny+\f@size@normalsize)/2}}
	{\fpeval{(\f@baselineskip@tiny+\f@baselineskip@normalsize)/2}}%
	\selectfont
}
\makeatother
\makeatletter
{\tiny % Capture font definitions of \small
	\xdef\f@size@tiny{\f@size}
	\xdef\f@baselineskip@tiny{\f@baselineskip}
	\normalsize % Capture font definitions for \normalsize
	\xdef\f@size@normalsize{\f@size}
	\xdef\f@baselineskip@normalsize{\f@baselineskip}
}
% Define new \smalltonormalsize font size
\newcommand{\othertinytonormal}{%
	\fontsize
	{\fpeval{(\f@size@tiny+\f@size@normalsize)*0.4}}
	{\fpeval{(\f@baselineskip@tiny+\f@baselineskip@normalsize)*0.4}}%
	\selectfont
}
\makeatother

% Headings
\usepackage{textcase}
\usepackage{mfirstuc}
\usepackage{titlecaps}
\usepackage{stringstrings}
\usepackage[raggedright,rm,bf]{titlesec}
\usepackage{titlecaps}
\titlelabel{\thetitle.\quad}
\titleformat{\chapter}[display]{\LARGE\rmfamily\raggedleft}{\scshape \chaptertitlename\ \thechapter}{3pt}{\titlerule\vspace{1.5ex} \scshape \Huge \centering}[\vspace{1.5ex}\titlerule]
\titlespacing*{\chapter}{0pt}{30pt}{40pt}  % remove spacing before chapter headings

\titleformat{\section}{\rmfamily\scshape\Large}{\thesection}{1em}{}
\titleformat{\subsection}{\rmfamily\scshape\large}{\thesubsection}{1em}{}
\titleformat{\subsubsection}{\rmfamily\scshape\normalsize}{\thesubsubsection}{1em}{}

\newcommand{\ShortInTextTitle}[1]{\textbf{\scshape #1}}

% Table of contents
\usepackage{xcolor,titletoc}
\makeatletter
\let\originall@chapter\l@chapter
\def\l@chapter#1#2{\originall@chapter{{\rmfamily #1}}{#2}}
\makeatother
\let \savenumberline \numberline
\def \numberline#1{\savenumberline{#1.}}

% Nomenclaure
\usepackage{acro}
\usepackage{multicol}
%\usepackage{scrextend}
\usepackage{ragged2e}
\newlength\myitemwidth
\setlength\myitemwidth{5cm}
\newlist{myacronymlist}{description}{1}
\setlist[myacronymlist]{
	%	itemsep = 1000pt,
	labelindent = 0pt ,
	labelsep    = 0pt ,
	leftmargin  = \myitemwidth ,
	labelwidth  = \myitemwidth ,
	itemindent  = 0pt ,
	format      = \RaggedRight 
}
%\NewAcroTemplate[list]{styleabbrev}{%
	%	\UseAcroTemplate[list]{description}[0]%
	%}
%\NewAcroTemplate[list]{styleabbrev}{%
	%		\UseAcroTemplate[list]{myacronymlist}[0]%
	%		\acropreamble
	%		\begin{itemize}
		%			\acronymsmapF
		%			{\item[\bfseries\acrowrite{short}] \acrowrite{alt}}
		%			{\item\AcroRerun}
		%		\end{itemize}
	%}
\RenewAcroTemplate{long-short}{%
	\acroiffirstTF{%
		\acrowrite{long}%
		\acspace(%
		\acroifT{foreign}{\acrowrite{foreign}, }%
		\acrowrite{short}%
		\acroifT{alt}{}%
		\acrogroupcite
		)%
	}%
	{\acrowrite{short}}%
}
\NewAcroTemplate[list]{styleabbrev}{%
	\normalsize
	\AcroNeedPackage{array}%
	\acronymsmapF{
		\AcroAddRow{
			\vspace{1.0ex}\par
			\acrowrite{short} & \acrowrite{alt}
			\acropagefill
			\acropages
			{\acrotranslate{page}\nobreakspace}
			{\acrotranslate{pages}\nobreakspace}%
			\tabularnewline
			% 			\vspace{1.8ex}\par
		}%
	}
	{\AcroRerun}%
	%    \acroheading
	\acropreamble
	\par\noindent
	\begin{tabular}{>{\bfseries}lp{.7\linewidth}}
		\AcronymTable
	\end{tabular}
}

\acsetup{
	list/template  = styleabbrev,
	%	list/display=used,
	%	list/sort=true,
	%	format=\RaggedRight
}

\usepackage[toc,section=section,nonumberlist]{glossaries-extra}
\usepackage{glossary-superragged}
\usepackage[intoc, english]{nomencl}
\usepackage{tabularx, booktabs}
\usepackage{calc}

\newlength{\mycolwidth}

%\newglossarystyle{mystyle}{
	%	\setglossarystyle{treegroup}%
	%	\renewcommand*{\glossentry}[2]{%
		%		\glsentryitem{##1}%
		%		\settowidth{\mycolwidth}{{\textsc{\mdseries\glossentryname{##1}} }}%
		%		\begin{tabular}{@{} l p{\dimexpr\textwidth-\mycolwidth\relax} @{}}
			%			{\textsc{\mdseries\glossentryname{##1}} }& \glossentrydesc{##1}\\
			%		\end{tabular}%
		%		\par\vspace{10pt}
		%	}
	%}
\newglossarystyle{mystyle}{
	\setglossarystyle{treegroup}%
	\renewcommand*{\glossentry}[2]{%
		\glsentryitem{##1}%
		%		\settowidth{\mycolwidth}{{\textsc{\mdseries\glossentryname{##1}} }}%
		{\textsc{\textbf{\glossentryname{##1}}}}
		\glossentrydesc{##1}
		\vspace{10pt}
		
	}
}
%\renewcommand{\glsnamefont}[1]{\textsc{\mdseries #1}}
\setglossarystyle{mystyle}

\renewcommand{\nomgroup}[1]{%
	\ifthenelse{\equal{#1}{V}}{\item[{\rmfamily\scshape\Large Variables}]}{%
		\ifthenelse{\equal{#1}{C}}{\item[{\rmfamily\scshape\Large Constants}]}{
			\ifthenelse{\equal{#1}{F}}{\item[{\rmfamily\scshape\Large Functions}]}{}}}%
}

\makenomenclature

\makeatletter
\renewcommand\tableofcontents{%
	\chapter*{\contentsname}%
	\@mkboth{\scshape \contentsname}%
	{\scshape \contentsname}%
	\@starttoc{toc}%
}
\renewcommand\listoftables{%
	\chapter*{\listtablename}%
	\@mkboth{\scshape\listtablename}%
	{\scshape\listtablename}%
	\@starttoc{lot}%
}
\renewcommand\listoffigures{%
	\chapter*{\listfigurename}%
	\@mkboth{\scshape\listfigurename}%
	{\scshape\listfigurename}%
	\@starttoc{lof}%
}
\makeatother


% Mathematics
%\usepackage[cmex10]{amsmath}
\usepackage{amssymb}
\usepackage{cancel}
\DeclareMathOperator*{\argmax}{arg\,max}
\newcommand{\T}{^\textrm{T}}
\newcommand{\tr}{\textrm{tr}}
\newcommand{\defeq}{\triangleq}

% Tables
\usepackage{placeins}
\usepackage{array}
\usepackage{colortbl}
\usepackage{longtable}
\usepackage{ltcaption}
%\renewcommand\LTcaptype{figure}
\usepackage{multirow}
\newcommand{\mytable}{
	\centering
	\small
	%     \renewcommand{\arraystretch}{1.2}
	\renewcommand{\arraystretch}{1.2}
}
\renewcommand{\tabularxcolumn}[1]{m{#1}}
\newcommand{\PreserveBackslash}[1]{\let\temp=\\#1\let\\=\temp}
\newcolumntype{C}{>{\centering\arraybackslash}X}
\newcolumntype{L}{>{\raggedright\arraybackslash}X}
\newcolumntype{A}[1]{>{\PreserveBackslash\raggedright}p{#1}}
\renewcommand{\topfraction}{.60}
\renewcommand{\bottomfraction}{.85}
\renewcommand{\textfraction}{.05}
\renewcommand{\floatpagefraction}{.66}
\renewcommand{\dbltopfraction}{.66}
\renewcommand{\dblfloatpagefraction}{.66}
\setcounter{topnumber}{2}
\setcounter{bottomnumber}{9}
\setcounter{totalnumber}{20}
\setcounter{dbltopnumber}{2}

% Descriptions
\usepackage[T1]{fontenc}
\usepackage{lmodern} % To switch to Latin Modern
\rmfamily % To load Latin Modern Roman and enable the following NFSS declarations.
% Declare that Latin Modern Roman (lmr) should take
% its bold (b) and bold extended (bx) weight, and small capital (sc) shape, 
% from the corresponding Computer Modern Roman (cmr) font, for the T1 font encoding.
\DeclareFontShape{T1}{lmr}{b}{sc}{<->ssub*cmr/bx/sc}{}
\DeclareFontShape{T1}{lmr}{bx}{sc}{<->ssub*cmr/bx/sc}{}
\setlist[description]{ 
	labelwidth=0.6\textwidth,%
	leftmargin=\labelwidth, 
	align=right,
	font=\normalfont\scshape\bfseries
}
\newcommand{\mydefinition}[2] {
	\begin{flushright}
		\textbf{\textsc{#1}} #2 \\
	\end{flushright}
}

\usepackage{makecell}
\usepackage{mdframed}
\newcommand{\defblock}[3]{
	\begin{mdframed}[roundcorner=100pt]
		\begin{tabular}{@{}lp{#3ex}@{}}
			\textbf{\textsc{#1}}: & #2
		\end{tabular}
	\end{mdframed}
}

\usepackage{environ}
\usepackage{tikz}
\usetikzlibrary{fit,backgrounds,calc}
% Pseudo-code
\usepackage{algorithm}  % should go before \usepackage{hyperref}
\NewEnviron{defbox}[3]
{
	%	\begin{tikzpicture}
		%		\node[inner sep=0pt,
		%		draw=cambrigeblue,
		%		line width=1.2pt,
		%		rounded corners=0.25cm] (box) {\parbox[t]{\textwidth}
			%			{
				\vspace{15pt}\\
				\begin{tabular}{@{}lp{#3ex}@{}}
					\RaggedLeft{\textbf{\textsc{#1}}:} & #2
				\end{tabular}
				\vspace{15pt}\\
				%			}%
			%		};
		%		
		%	\end{tikzpicture}
}


% Table of contents and hyperlinks
\usepackage[linktocpage=true]{hyperref}
%\hypersetup{colorlinks=true,linktoc=all,allcolors=gray!80!black}
\hypersetup{
	colorlinks=true,
	linktoc=all,
	linkcolor=black, 
	filecolor=black, 
	citecolor = gray!80!black,       
	urlcolor=gray!80!black,
}
%\usepackage[nottoc,notlot,notlof]{tocbibind}
\usepackage{blindtext}

% Pseudo-code
\usepackage{algpseudocode}  % should go after \usepackage{hyperref}
\renewcommand{\thealgorithm}{\arabic{chapter}.\arabic{algorithm}} 
\captionsetup[algorithm]{labelfont={bf},font=small,labelsep=colon}

% Fix titlesec issue
\usepackage{etoolbox}
\makeatletter
\patchcmd{\ttlh@hang}{\parindent\z@}{\parindent\z@\leavevmode}{}{}
\patchcmd{\ttlh@hang}{\noindent}{}{}{}
\makeatother

% Bibliography
%\usepackage[square,numbers,compress,sort]{natbib}
%\bibliographystyle{IEEEtranN}
\usepackage[round]{natbib}
%\bibliographystyle{apalike}
%\usepackage{cite}
\usepackage{bibentry}
\nobibliography*
%\newcommand{\mycite}[1]{\citeauthor{#1}'s~\cite{#1}}

% Header and footer
\newcommand{\listheaderfont}{\scshape}
%\renewcommand{\tocetcmark}[1]{
	%	\markboth{\listheaderfont #1}{\listheaderfont #1}}
\usepackage{fancyhdr}
\usepackage{extramarks}
\pagestyle{fancy}
\fancyhf{}
\renewcommand{\chaptermark}[1]{\markboth{\normalsize \rmfamily \thechapter.\ \scshape #1}{\normalsize \rmfamily \thechapter.\ \scshape #1}}
\renewcommand{\sectionmark}[1]{\markright{\normalsize \rmfamily \thesection.\ \scshape #1}}
%\renewcommand{\sectionmark}[1]{\markright{\normalsize \rmfamily \thesection.\ \scshape #1}{}}
\fancyhead[C]{\color{oraclecolor} \rightmark \\ \color{rulecolor} \rule{\textwidth}{0.1pt}}
\fancyfoot{}
\fancyfoot[R]{\thepage}{}

\setlength\headheight{14.5pt}
\renewcommand{\headrulewidth}{0pt}
\fancypagestyle{plain}{\fancyhead{}
	\renewcommand{\headrulewidth}{0pt}
	%	\fancyfoot[C]{\thepage}
}

\newcommand{\mysubsubsection}[2]{
	%	\sectionmark{#2}
	\let\orisubsubsectionmark\subsubsectionmark
	\renewcommand\subsubsectionmark[1]{}%
	\subsubsection[#1]{#2 \orisubsubsectionmark{#2}}
	\orisubsubsectionmark{#2}
	\let\subsubsectionmark\orisubsubsectionmark
}
\newcommand{\mysubsection}[2]{
	%	\sectionmark{#2}
	\let\orisubsectionmark\subsectionmark
	\renewcommand\subsectionmark[1]{}%
	\subsection[#1]{#2 \orisubsectionmark{#2}}
	\orisubsectionmark{#2}
	\let\subsectionmark\orisubsectionmark
}
\newcommand{\mysection}[2]{
	%	\sectionmark{#2}
	\let\orisectionmark\sectionmark
	\renewcommand\sectionmark[1]{}%
	\section[#1]{#2 \orisectionmark{#2}}
	\orisectionmark{#2}
	\let\sectionmark\orisectionmark
}
\newcommand{\mychapter}[3]{
	%	\sectionmark{#2}
	\chapter[#1]{#2 \chaptermark{#2}}\vspace{-15pt}\begin{center}
		{\Large\scshape #3} 
	\end{center} 
	\vspace{10pt}
}

% Miscellaneous
\interfootnotelinepenalty=1000
\newcommand*{\WaterMark}[2][0.2\paperwidth]{\AddToShipoutPicture*{\AtTextCenter{\parbox[c]{0pt}{\makebox[0pt][c]{\includegraphics[width=#1]{#2}}}}}}

% Colors and editing
\usepackage[prependcaption,textsize=tiny]{todonotes}
\setlength{\marginparwidth}{2.6cm}
\reversemarginpar
\definecolor{othercolor}{HTML}{CC0000}
\definecolor{oraclecolor}{HTML}{8B8589}
\definecolor{rulecolor}{HTML}{DCDCDC}
\definecolor{ashgrey}{rgb}{0.7, 0.75, 0.71}
\newcommand{\change}[1]{\textcolor{othercolor}{#1}}

\usepackage{pifont}
\usepackage{xcolor}
\definecolor{mycolor}{HTML}{008000}%{FF6600}
\definecolor{indiagreen}{HTML}{138808}%{FF6600}
\definecolor{papaya}{HTML}{EE892F}%{FF6600}
\definecolor{mygreen}{HTML}{008000}%{FF6600}
\definecolor{mypurple}{HTML}{9966CC}%{FF6600}
\definecolor{cambrigeblue}{HTML}{A3C1AD}%{FF6600}
\definecolor{darkgreen}{HTML}{014421}
\definecolor{lightgreen}{HTML}{8FBC8F}


\NewEnviron{publicationinfo}[1]
{
	\begin{tikzpicture}
		\node[inner sep=0pt,
		draw=black!50!white,
		line width=2.5pt,
		fill=black!10!white,
		%		rounded corners=0.25cm
		] (box) {\parbox[t]{\textwidth}
			{
				\begin{center}
					\begin{minipage}{0.9\textwidth}
						%					\vskip 10pt
						%					\textbf{#1}\par\smallskip
						{\BODY}
						\par\smallskip
					\end{minipage}\hfill
				\end{center}
			}%
		};
		\par
	\end{tikzpicture}
}

\NewEnviron{researchquestions}[1]
{
	\begin{tikzpicture}
		\node[inner sep=0pt,
		draw=black!50!white,
		line width=1.0pt,
		fill=black!10!white,
		%		rounded corners=0.25cm
		] (box) {\parbox[t]{\textwidth}
			{
				\begin{center}
					\begin{minipage}{0.9\textwidth}
						%					\vskip 10pt
						%					\textbf{#1}\par\smallskip
						{\BODY}
						\par\smallskip
					\end{minipage}\hfill
				\end{center}
			}%
		};
		\par
	\end{tikzpicture}
}

\NewEnviron{contributionninfo}[1]
{
	\begin{tikzpicture}
		\node[inner sep=0pt,
		draw=darkgreen!50!white,
		line width=2.5pt,
		fill=darkgreen!20!white,
		%		rounded corners=0.25cm
		] (box) {\parbox[t]{\textwidth}
			{
				\begin{center}
					\begin{minipage}{0.9\textwidth}
						%					\vskip 10pt
						%					\textbf{#1}\par\smallskip
						{\BODY}
						\par\smallskip
					\end{minipage}\hfill
				\end{center}
			}%
		};
		\par
	\end{tikzpicture}
}

\usepackage{tipa}
\usepackage{combelow}

%{\scshape \chaptertitlename\ \thechapter}{3pt}{\titlerule\vspace{1.5ex} \scshape \Huge \centering}[\vspace{1.5ex}\titlerule]

\newcommand{\divider}[1]{
	\newpage
	\vspace*{\fill}
	\titlerule\vspace{5ex}
	\begin{spacing}{2.5}
		\begin{center}
			{\Huge\scshape #1}
		\end{center}
	\end{spacing}
	\thispagestyle{empty}
	\vspace{1ex}
	\titlerule
	\vspace*{\fill}
	\addtocounter{page}{-1}
}

\usepackage{cleveref}
\crefname{figure}{Figures}{Figures}
\Crefname{figure}{Figures}{Figures}
%\usepackage{natbib}
%\bibliographystyle{apalikefull}
%\usepackage{apalike}