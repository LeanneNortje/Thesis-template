\chapter*{Nomenclature\markboth{}{Nomenclature}}
\addcontentsline{toc}{chapter}{Nomenclature}

% \vspace*{-3mm}
\section*{Variables and functions}
\markboth{}{\scshape \nomname}
\setlength{\nomlabelwidth}{3cm}
\printnomenclature

%\addcontentsline{toc}{section}{Variables}
%\nomenclature[v]{$K$}{The amount examples per class in the few-shot learning setting.}
%\nomenclature[v]{\supportSet}{The set of given examples per class from which the model should learn new classes.}
%\nomenclature[v]{$L$}{The number of classes that should be learned in the few-shot learning setting.}
%\nomenclature[v]{\querySet}{The set of queries that has to be classified by the few-shot model.}
%\nomenclature[v]{\matchingSet}{The set of images from which the model should choose an image that matches a spoken word query.}
\newpage
\section*{Acronyms and Abbreviations}
\addcontentsline{toc}{section}{Acronyms and abbreviations}

\printacronyms[display=used]

\glsaddall
\newpage
\renewcommand\glossaryname{Definitions}
\printnoidxglossaries

%\newglossarystyle{mystyle}{%
	%%	\setglossarystyle{treegroup}%
	%	\renewcommand*{\glossentry}[2]{%
		%		\glsentryitem{##1}%\textit{\glstarget{##1}{\glossentryname{##1}}}\par
		%	}%
	%}
%
%\setglossarystyle{mystyle}

\newpage